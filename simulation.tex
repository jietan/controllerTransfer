\section{Physical Simulation}

\subsection{Dynamics Equation}

We model the robot as an articulated rigid body system in our simulator, which satisfies the following dynamics equations, non-penetration and linear complementarity conditions for contact points.

\begin{align}
\label{eq:robotdynamics}
\mathbf{M}(\mathbf{q})\mathbf{\ddot{q}}+\mathbf{C}(\mathbf{q},\mathbf{\dot{q}})&=\boldsymbol{\tau}+\mathbf{J}(\mathbf{q})^T\mathbf{f}\\
\nonumber \mathbf{d}(\mathbf{q})&\geq \mathbf{0}\\
\nonumber \mathbf{d}(\mathbf{q})^T\mathbf{f}&= \mathbf{0}\\
\nonumber \mathbf{f}&\in\mathbf{K}
\nonumber \end{align}

where $\mathbf{M}(\mathbf{q})$ is the mass matrix and $\mathbf{C}(\mathbf{q},\mathbf{\dot{q}})$ is the Coriolis and Centrifugal force. $\boldsymbol{\tau}$ are joint torques exerted by the actuators. An actuator model that computes $\boldsymbol{\tau}$ based on the current state and a reference pose is presented in the next section. $\mathbf{J}(\mathbf{q})$ is the Jacobian matrix and $\mathbf{f}$ is the contact force. $\mathbf{d}(\mathbf{q})$ is the distance of the contact to the ground and $\mathbf{K}$ is the friction cone. We use DART to solve the above equations to simulate the dynamics of the robot.

\subsection{Actuator Model}
\label{sec:motorDynamics}
A faithful robot simulation relies heavily on an accurate actuator model. A common practice is to set the same PID gains in the simulation as those in the servo. However, Dynamixel AX-18 servos, used on our robot (ROBOTIS BIOLOID GP), do not support PID control. Although the servo can track an input reference angle $\bar{q}$, the relation between the joint error $\bar{q}-q$ and the output torque $\tau$ is unclear. We derive an actuator model based on the ideal DC motor assumption and the specification of the servo:

\begin{equation}
  \tau = -k_p(q-\bar{q}) - k_d\dot{q} - k_c\sgn(\dot{q})\\
    \label{eqn:torqueErrorRelationSimple}
\end{equation}
where $k_p$, $k_d$ and $k_c$ the \emph{actuator gains}. The detailed derivation of the actuator model (eq. (\ref{eqn:torqueErrorRelationSimple})) can be found in Appendix.

Accurately identifying these gains would require a large number of carefully designed experiments to be run on each of the actuators. In contrast, we choose to run only one simple experiment on one actuator. This simple experiment gives us an initial guess of the gains. We then rely on simulation calibration to refine this estimation.

In this experiment, we clamp the entire robot on a table except for the left foot. We then send a periodic reference joint angle $\bar{q}(t)$ that oscillates between two extreme angles (blue curve in Figure \ref{fig:actuatorId}) to the servo at the left ankle. We record the trajectory of the actual joint angle $q(t)$ throughout the experiment (green curve in Figure \ref{fig:actuatorId}). 

\begin{figure}[!t]
  \centering
  \includegraphics[width=0.3\textwidth]{figures/actuatorId}
  \caption{The trajectories of the reference and the measured joint angle for acutator identification.}
  \vspace{-0.1in}
  \label{fig:actuatorId}
\end{figure}

Given $q(t)$ and $\bar{q}(t)$, we apply regression to estimate the actuator gains:

\begin{displaymath}
\min_{k_p, k_d, k_c}\int||I\ddot{q}(t)+k_p(q(t)-\bar{q}(t)) + k_d\dot{q}(t) + k_c\sgn(\dot{q}(t))||^2\mathrm{d}t
\end{displaymath}
where $I$ is the moment of inertia of the foot with respect to the rotating axis. The above equation is derived by plugging into $\tau = I\ddot{q}+\dot{I}\dot{q}$ and the fact that $\dot{I}\dot{q}=0$ because the foot is a rigid body that rotates along a fixed axis. Our experiments and computation show that the actuator gains are $k_p=9.272(N\cdot m/rad)$, $k_d=0.3069(N\cdot m\cdot s/rad)$, and $k_c=0.03(N\cdot m)$. 



%\begin{displaymath}
%\min_{\boldsymbol{\theta}}\frac{1}{n}\sum_{i=1}^{n}\int||\tilde{\mathbf{x}}_i(\bar{\mathbf{q}}(t))-\mathbf{x}(\bar{\mathbf{q}}(t);\boldsymbol{\theta})||^2\mathrm{d}t
%\end{displaymath}
