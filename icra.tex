%%%%%%%%%%%%%%%%%%%%%%%%%%%%%%%%%%%%%%%%%%%%%%%%%%%%%%%%%%%%%%%%%%%%%%%%%%%%%%%%
%2345678901234567890123456789012345678901234567890123456789012345678901234567890
%        1         2         3         4         5         6         7         8

\documentclass[letterpaper, 10 pt, conference]{ieeeconf}  % Comment this line out if you need a4paper

%\documentclass[a4paper, 10pt, conference]{ieeeconf}      % Use this line for a4 paper

\IEEEoverridecommandlockouts                              % This command is only needed if 
                                                          % you want to use the \thanks command

\overrideIEEEmargins                                      % Needed to meet printer requirements.

% See the \addtolength command later in the file to balance the column lengths
% on the last page of the document

% The following packages can be found on http:\\www.ctan.org
\usepackage{graphics} % for pdf, bitmapped graphics files
\usepackage{epstopdf}
\usepackage{epsfig} % for postscript graphics files
%\usepackage{mathptmx} % assumes new font selection scheme installed
%\usepackage{times} % assumes new font selection scheme installed
\usepackage{amsmath} % assumes amsmath package installed
\usepackage{amssymb}  % assumes amsmath package installed
\DeclareMathOperator{\sgn}{sgn}
\title{\LARGE \bf
Simulation-Driven Robot Controller Design
}


\author{Jie Tan$^{1}$ ~~~Zhaoming Xie$^{1}$ ~~~Byron Boots$^{2}$ ~~~Karen Liu$^{2}$% <-this % stops a space
\thanks{$^{1}$The authors are with College of Computing, Georgia Institute of Technology, Atlanta, GA, USA. Email:
        {\tt\small \{jtan34, zxie47\}@gatech.edu}}%
\thanks{$^{2}$The authors are with College of Computing, Georgia Institute of Technology, Atlanta, GA, USA. Email:
        {\tt\small \{bboots, turk, karenliu\}@cc.gatech.edu}}%
}


\begin{document}

\maketitle
\thispagestyle{empty}
\pagestyle{empty}


%%%%%%%%%%%%%%%%%%%%%%%%%%%%%%%%%%%%%%%%%%%%%%%%%%%%%%%%%%%%%%%%%%%%%%%%%%%%%%%%
\begin{abstract}
Many of our daily activities, such as sit-to-stand, involve changing postures while maintaining balance. Understanding and synthesizing these transition motions can have fundamental impacts in robotics, but controlling a humanoid robot to perform such motions is challenging. In this paper, we present a system that can automatically design robot controllers for these transition tasks. Our system uses simulation and optimization algorithms to find optimal controllers in a physically-simulated environment. It then relies on simulation calibration to transfer the controller from the virtual to the real world. Our experiments show that this system can automatically find successful controllers for a wide variety of transition tasks, including lean-to-stand, sit-to-stand, kneel-to-stand and stand-to-handstand.

\end{abstract}

\section{Introduction}

\ignorethis{\karen{Need to emphasize ``dynamic balance''. Mention the challenges of these tasks in this paragraph.}}

Many of our daily activities involve changing our posture to a target pose without losing balance. For example, to stand up from a chair, we carefully maintain balance while switching from a sitting pose to a standing pose. There are a large variety of motions of similar kind, such as lean-to-stand, kneel-to-stand and stand-to-handstand. Although these tasks can be achieved with slow motions and static balance, we human can perform them in an agile fashion. This agility comes from our ability to control \emph{dynamic balance}: We can use rapid movements to maintain balance even when our center of mass (COM) is outside the support polygon. Studying these motions can have far-reaching impacts in robotics, especially in rehabilitation, exoskeleton and humanoid robots. However, understanding and synthesizing these motions are extremely challenging because they involve complex motor skills, delicate balance strategies and rich interactions with the environments. 

\ignorethis{\karen{Admit that this can be done and has been done through manual tuning, but emphasize that we will achieve this by automatically transfer a simulated controller to the real world. Mention the challenges of reality gap.}}


The goal of this paper is to present a system to automatically design humanoid robot controllers for these locomotion tasks. We aim at the controllers that can demonstrate comparable agility to its human counterparts. Although it is possible to design these controllers through laborious manual tuning and numerous costly robot experiments, our system is automatic and requires minimal human intervention. It replaces the tedious manual tuning with automatic optimization processes, and drastically reduces the number of expensive robot experiments. Our system consists of three main components, physical simulation, controller optimization and simulation calibration. We first build a \emph{physical simulation} to simulate the dynamics of the robot and its interaction with the environments. We use \emph{controller optimization} to search for the optimal joint trajectories to fulfil the task in the simulation. However, even though this optimal controller works effectively in the simulation, it can perform poorly on a robot in the real environment. This ``Reality Gap'' \cite{Jakobi95} is caused by various simplifications in simulation algorithms, such as simplified actuator models, inaccurate physical parameters, and ignored hardware limitations, noise and latency. 

\ignorethis{\karen{Describe our idea and how it solve the reality gap.}}

We use \emph{simulation calibration} to cross the Reality Gap. Simulation calibration is a dynamic system identification method. We want to emphasize that the goal of simulation calibration is to effectively design controllers that works in the real world, instead of finding the ground truth about the hardware parameters through extensive experiments. During calibration, we collect real performance data on the robot, and use it to improve our physical model. Unlike traditional system identification, which is a separate step from controller design, our simulation calibration is tightly coupled with the controller optimization. We directly perform the optimal controller that is found in the simulation to collect real data for system identification. In this way, the simulation is calibrated at the vicinity of the current optimal controller. The computation is focused at the important regions of the control space and thus fewer robot experiments are needed. Through calibration, the simulator can capture the real world dynamics more faithfully. This calibrated simulator is used again in controller optimization to improve the quality of the controller. We evaluate our system using four locomotion tasks: lean-to-stand, sit-to-stand, kneel-to-stand and stand-to-handstand. Although these tasks are distinctively different, our system can find successful controllers for all the tasks efficiently and automatically. 

The main contribution of this paper is a complete pipeline that can automatically design robot controllers for a wide range of locomotion tasks. Starting from task specifications, our system can find controllers that operate on the robot, with minimal human intervention. This simulation-driven approach can effectively cut the time and the cost of designing robot controllers. 

\section{Related Work}

The locomotion tasks that this paper foucses on involve changing between two poses and maintaining balance during the motion. Two tasks that falls into this category and are extensively studied in robotics are sit-to-stand \cite{Faloutsos:2003,Iida:2004,Pchelkin:2010,Mistry:2010,Bahar:2014} and lie-to-stand \cite{morimoto:1998,Faloutsos:2001,Kanehiro:2007,Hirukawa:2005}. While many of these prior work focuses on one specific motion, we target at a more general problem of finding controllers for a wide range of such locomotion tasks. A few related work tried to tackle this more general problem. Jones \cite{jones:2011} developed rising motions for both biped and quadraped using pose tracking, orientation correction and virtual force. Lin and Huang \cite{lin:2012} used motion planning and dynamics filtering to develop rising up motions from various initial lying poses. Tassa et al. \cite{tassa:2012} used Model Predictive Control to synthesize complex behaviors, including getting up from an aribitrary pose on the ground. Although these work has shown impressive results in simulation, experiments on real robots were not presented.

Locomotion often involve frequent change of contacts. It poses significant challenges to controller optimization due to the discontinuous contact forces. We choose to use Covariance Matrix Adaptation (CMA) \cite{Hansen:2004}, a stochastic sampling-based optimization algorithm, to tackle this challenge. Although CMA is not widely used in robotics, it is a popular method in physically-based character animation to search for control parameters when the problem domain is highly discontinous \cite{Wu:2010, Wang:2010, Tan:2011}.

A controller that is designed in simulation may not work in the real environment. This discrepancy is called Reality Gap. One way to cross the reality gap is to improve the simulation model using real data measured from robot experiments. 

system identification need a lot of data because the task used in system identification is often different from the real task. Real task involves contacts, which makes system identification difficult. CMA is used to optimize gait in computer animation, which should be able 

. The simulation is improved by measuring and minimizing the discrepancy between the simulation results and the data collected in robot experiments. Ha and Yamane \cite{HA:2015} modeled this discrepancy using Gaussian process. Abbeel et al. \cite{Abbeel:2006} used an inaccurate physical model but successively grounded the policy evaluations using real-life trials.Grounded simulated learning approach \cite{Farchy:2013} iteratively optimized the controller, measured the discrepancy and modified the simulator using supervised learning algorithms. Bongard and Lipson \cite{BongardL05} coevolved the controller and the simulator using an iterative estimation-exploration process. Similarly, Zagal et al. \cite{zagal2004} introduced the ``back-to-reality'' approach, which also involved the coevolution but used a different measure of discrepancy.

\section{Overview}

We have developed a system that can automatically design locomotion controllers for robots (Figure~\ref{fig:controllerTransferOverview}). Given the specification of the robot, including its body shape, the physical properties of each body, and the types of joints, we build a physical simulation using Dynamic Animation and Robotics Toolkit (DART) \cite{dart:2012}. In addition, we also incorporate into the simulation the torque limits, servo models and communication latency, which are often omitted in character animation. The controller optimization subsystem runs thousands of simulations to search for the optimal controller that maximizes the task-related fitness function. We then test this optimal controller on the robot. If the robot successfully completes the task, a working robotic controller is found and our algorithm terminates. Otherwise, we record the robot performance data and feed it into the simulation calibration subsystem. Simulation calibration runs another optimization, which searches for the optimal simulation parameters to minimize the discrepancy between the performance of the robot in the simulation and in the real world. The loop of controller optimization and simulation calibration is performed iteratively until the controller works successfully on the real robot. In the next three sections, we will present the algorithmic details of these components.

\section{Physical Simulation}

\subsection{Dynamics Equations}

We model the robot as an articulated rigid body system in our simulator. We represent the states of the system $(\mathbf{x}, \dot{\mathbf{x}})$ in the generalized coordinate, where $\mathbf{x}$ include the global position $\mathbf{p}$, orientation $\mathbf{r}$ of the root link, and the joint angles $\mathbf{q}$. We solve the governing equations of motion eq.(\ref{eq:robotdynamics}) in the generalized coordinates.

\begin{equation}
\label{eq:robotdynamics}
\mathbf{M}(\mathbf{x})\mathbf{\ddot{x}}+\mathbf{C}(\mathbf{x},\mathbf{\dot{x}})=\mathbf{\tau}+\mathbf{J}^T\mathbf{f}
\end{equation}
where $\mathbf{M}(\mathbf{x})$ is the mass matrix and $\mathbf{C}(\mathbf{x},\mathbf{\dot{x}})$ is the Coriolis and Centrifugal force. $\mathbf{\tau}$ are joint torques exerted by the actuators. $\mathbf{J}$ is the Jacobian matrix and $\mathbf{f}$ is the external contact force, which is computed based on linear complementarity conditions. In our implementation, we use DART to compute the contact force and numerically integrate the system state $(\mathbf{x}, \dot{\mathbf{x}})$ over time.

\subsection{Actuator Model}
\label{sec:motorDynamics}
In character animation, the joint torque $\tau$ is often chosen as the control signal since they can be directly integrated in eq.(\ref{eq:robotdynamics}). However, the control signal for the robot that we use in the experiments is the desired joint angles $\bar{\mathbf{q}}$. Given the difference between the desired and the current angle ${q-\bar{q}}$ of each joint, the servo first maps it to a corresponding power level $U$ that is equivalent to changing the voltage across the motor and the voltage is eventually converted to the joint torque $\tau$ according to the internal actuator dynamics.

\begin{figure}[t]
\centering
\includegraphics[width=0.45\textwidth]{figures/ax18gain.eps}
\caption{The mapping between $q-\bar{q}$ and $U$ for an AX-18 actuator \cite{AX18:2015}. The x-axis is $q-\bar{q}$ while the y-axis is $U$.}
\label{fig:actuatorMap}
\end{figure}

Most of the actuators on our robot are Dynamixel AX-18, which use the following mapping (Figure \ref{fig:actuatorMap}) between the joint angle difference $q-\bar{q}$ and the power level $U$. The intervals A and D determine the slope of the actuator response for counter-clock-wise and clock-wise motions respectively. Smaller values mean steeper response slopes, in which case the actuator follows the desired angle more closely. However, too small values can lead to overshooting problems. B and C are the compliance margins. If the error of angle is within a small margin specified by B and C, the servo does not output any torque. E, the punch, is the minimum power level before the servo shuts down. In practice, we set A and D to be the same so that the servo will behave the same no matter it rotates clock-wise or counter-clock-wise. In addition, since B, C and E are very small compared to A and D, we ignore their effects and approximate the mapping as linear within the intevals $q-\bar{q}\in A\bigcup B\bigcup C\bigcup D$ with the slope $k_e$:
\begin{equation}
  U=k_e(q-\bar{q})
  \label{eqn:voltageErrorRelation}
\end{equation}

To derive the relation between the power level $U$ and the output torque $\tau$, we adopt a model for the ideal DC motor \cite{SchwarzB:2013}. It is valid to assume an ideal model because the AX-18 servos use high-quality DC motors. The derivation follows by considering the power balance in the motor at a constant voltage U:
\begin{equation}
  P_{electric} = P_{mechanic} + P_{heat}
  \label{eqn:powerBalance}
\end{equation}
where $P_{electric}$ is the electrical power, $P_{mechanic}$ is the mechanical power, and $P_{heat}$ is the power dissipated as heat. From eq.(\ref{eqn:powerBalance}), we can get the following relation:
\begin{equation}
UI=\dot{q}\tau_{motor} + RI^2
\end{equation}
where $I$ is the current and $R$ is the motor winding resistance. In an ideal DC motor, the torque is linearly proportional to the current $\tau_{motor}=k_{\tau}I$. Plugging it into the above equation, we arrive at the relation between the voltage $U$ and the total torque generated by the motor $\tau_{motor}$:
\begin{equation}
  U=k_{\tau}\dot{q}+\frac{R}{k_{\tau}}\tau_{motor}
  \label{eqn:votageTorqueRelation}
\end{equation}
where $k_{\tau}$ is the torque constant, which is determined by the hardware design of the motor. The total torque generated by the motor is not yet the output torque that drives the motor shaft due to the friction inside the motor. The total torque can be decomposed into the output torque $\tau$ and the friction torque $\tau_f$.
\begin{equation}
  \tau_{motor}=\tau+\tau_f
  \label{eqn:torqueBalance}
\end{equation}
The friction torque can be further divided into viscous friction and Coulomb friction \cite{SchwarzB:2013}:
\begin{equation}
  \tau_f = k_v\dot{q}+k_c\sgn(\dot{q})
  \label{eqn:frictionComponents}
\end{equation}
where $k_v$ and $k_c$ are friction coefficients for the viscous and Coulomb friction respectively. $\sgn(x)$ is the sign function that equals 1 if $x$ is positive, -1 if $x$ is negative and 0 otherwise.

Combining eq.(\ref{eqn:votageTorqueRelation}), (\ref{eqn:torqueBalance}) and (\ref{eqn:frictionComponents}), we get the relation between the error of the joint angle $q-\bar{q}$ and the output torque $\tau$.
\begin{align}
\nonumber  \tau & = \frac{k_{\tau}k_e}{R}(q-\bar{q})+(-k_v-\frac{k_{\tau}^2}{R})\dot{q}-k_c\sgn(\dot{q})\\
\nonumber & = -k_p(q-\bar{q}) - k_d\dot{q} - k_c\sgn(\dot{q})\\
  \label{eqn:torqueErrorRelationSimple}
\end{align}
where $k_p=-\frac{k_{\tau}k_e}{R}$ and $k_d=k_v+\frac{k_{\tau}^2}{R}$. We call these values $k_p$, $k_d$ and $k_c$ the \emph{actuator gains}. It is possible to compute these actuator gains if the related parameters are given in the specification sheet of the motor. Plugging eq.(\ref{eqn:torqueErrorRelationSimple}) into (\ref{eq:robotdynamics}), and taking torque limits $[\tau_{min}, \tau_{max}]$ into consideration, we get the dynamics equation that use the desired joint angles as the control signal.
\begin{displaymath}
 \mathbf{M}(\mathbf{x})\mathbf{\ddot{x}}+\mathbf{C}(\mathbf{x},\mathbf{\dot{x}}) = \tau+\mathbf{J}^T\mathbf{f} \\
  \end{displaymath}
where 
\begin{displaymath}\tau =
  \left\{
    \begin{array}{ll}
      \tau_{min} & \text{if }\tau < \tau_{min},\\
      \tau_{max} & \text{if }\tau > \tau_{max},\\
      -k_p(\mathbf{q}-\bar{\mathbf{q}}) - k_d\dot{\mathbf{q}} - k_c\sgn(\dot{\mathbf{q}}) & \text{otherwise.}\\
    \end{array}
  \right.
  \label{eqn:robotDynamicsControl}
\end{displaymath}

\paragraph{Actuator Gain Identification.} We design robot experiments to identify the actuator gains $k_p$, $k_d$ and $k_c$, since the specification of the servos does not provide the necessary information to compute them. In the experiment, we clamp the entire robot on a table except for the left foot. We then send a periodic control signal $\bar{q}(t)$ to the servo at the left ankel (blue curve in Figure \ref{fig:actuatorId} Left). The desired joint angle stays at the maximum value for 0.67 second, then changes to the minimum value and stays for another 0.67 second and repeats. We record the trajectory of the actual joint angle $q(t)$ through the experiment (green curve in Figure \ref{fig:actuatorId} Left). We manually segment out portions of these two curves where the power level is approximately linear to the error $\Delta q = q-\bar{q}$ (the union of intervals A, B, C and D in Figure \ref{fig:actuatorMap}). The black ``+'' in Figure~\ref{fig:actuatorId} Right shows this error over time $\Delta q(t)$ in a typical segment.

\begin{figure}[!t]
  \centering
  \includegraphics[width=0.5\textwidth]{figures/actuatorId}
  \caption{Actuator Identification. Left: the time series of input desired joint angle and the measured joint angle for an AX-18 servo. Right: the time series of actual error of joint angle and the predicted error using the identified actuator gains.  }
  \label{fig:actuatorId}
\end{figure}

Given $q(t)$ and $\bar{q}(t)$, we can apply regression to estimate the actuator gains. From eq. (\ref{eqn:torqueErrorRelationSimple}), we have
\begin{equation}
\ddot{q}=I^{-1}(-k_p\Delta q - k_d\dot{q} - k_c\sgn(\dot{q}))
\end{equation}
where $I$ is the moment of inertia of the foot with respect to the rotating axis. The above equation is derived by plugging into $\tau = I\ddot{q}+\dot{I}\dot{q}$ and the fact that $\dot{I}\dot{q}=0$ because the foot is a rigid body that rotates along a fixed axis. Ideally, $\ddot{q}$ and $\dot{q}$ can be computed using finite difference. However, the measurement of $q(t)$ is too noisy and finite difference would greatly magnify the noise. To solve this problem, we first smooth $q(t)$ by performing a 4th-order polynomial regression:
\begin{equation}
  \min_{a,b,c,d,e}\int ||q(t)-(at^4+bt^3+ct^2+dt+e)||^2\mathrm{d}t
  \label{eqn:Deltaq}
\end{equation}
where $a,b,c,d,e$ are the polynomial coefficients. This regression gives us a smooth analytical expression of $q(t)$. We then compute $\ddot{q}$ and $\dot{q}$ by differentiate this polynomial analytically:
\begin{align}
\label{eqn:Deltaqdot}  \dot{q}(t)&=4at^3+3bt^2+2ct+d\\
\label{eqn:Deltaqddot}  \ddot{q}(t)&=12at^2+6bt+2c
\end{align}

Combining eq. (\ref{eqn:Deltaq}), (\ref{eqn:Deltaqdot}) and (\ref{eqn:Deltaqddot}), we can perform another regression to compute the actuator gains.
\begin{equation}
\min_{k_p, k_d, k_c}\int||\ddot{q}(t)-I^{-1}(-k_p(q(t)-\bar{q}(t)) - k_d\dot{q}(t) - k_c\sgn(\dot{q}(t)))||^2\mathrm{d}t
\end{equation}

Our experiments and computation show that the actuator gains are $k_p=9.272(N\cdot m/rad)$, $k_d=0.3069(N\cdot m\cdot s/rad)$, and $k_c=0.03(N\cdot m)$. To verify the correctness of these values, we plug them into the simulator and repeat the same experiment in the simulation. The red curve in Figure \ref{fig:actuatorId} Right is the error over time predicted in our simulation, which agrees well with the data that was collected from the robot experiment.

\paragraph{Latency.} To guarantee the stability of the simulation, we use 1ms as the simulation time step. Many animation systems use the same simulation and control frequency, which means that a control signal $\bar{\mathbf{q}}$ is updated every simulation time step. However, the average latency of the whole control loop on our robot is 16ms. It is measured by a timer in our program between the time that the program starts sending the actuator commands to the robot and the time that it finishes reading the sensor measurements from the robot. To better match our simulation with the actual latency, we choose to only update the control signal every 16 time steps.

\section{Controller Optimization}

Given the physical simulation, we can design controllers to enable the robot to achieve various locomotion tasks in the simulated environment. The three tasks that we use to test our system are rising from a leaning, sitting or kneeling position to an erect stance (Figure \ref{fig:task}). For each task, the joint configuration of the initial pose and the final pose are provided by the user. The goal of controller optimization is to find a sequence of control signals $\bar{q}(t)$ so that the robot can move from the initial to the final pose without losing balance. We purposefully choose to use only feedforward controllers\footnote{There is still an internal feedback loop in the actuators to track the desired joint angle (See Chapter~\ref{sec:motorDynamics}). However, this feedback loop is not fully programmable.} in this work, which means that the control signal $\bar{\mathbf{q}}(t)$ is a only function of time $t$ and does not depend on the states of the robot. With the feedforward control alone, the controller transfer can only succeed if the simulation is close enough to the real-world environment. This will put the simulation calibration subsystem into more thorough tests.

\begin{figure*}[!t]
  \centering
  \includegraphics[width=0.7\textwidth]{figures/initialFinal}
  \caption{The initial and the final joint configurations of the locomotion tasks. Top row: the initial poses of leaning, sitting and kneeling. Bottom row: the final standing pose for all three tasks.}
  \label{fig:task}
\end{figure*}


We first formulate a trajectory optimization problem for each task.
\begin{align}
 \label{eqn:obj}&\max_{\bar{\mathbf{q}}(t),T} V_{ctrl}(\mathbf{x}(t))\\
\nonumber  \mathrm{subject\;} &\mathrm{to} \\
\label{eqn:dyn1} & \mathbf{M}(\mathbf{x})\mathbf{\ddot{x}}+\mathbf{C}(\mathbf{x},\mathbf{\dot{x}}) =\tau + \mathbf{J}^T\mathbf{f}\\
\label{eqn:dyn2} &\tau =
  \left\{
    \begin{array}{ll}
      \tau_{min} & \text{if }\tau < \tau_{min},\\
      \tau_{max} & \text{if }\tau > \tau_{max},\\
      -k_p(\mathbf{q}-\bar{\mathbf{q}}) - k_d\dot{\mathbf{q}} - k_c\sgn(\dot{\mathbf{q}}) & \text{otherwise.}\\
    \end{array}
  \right.\\
\label{eqn:boundary1}&\bar{\mathbf{x}}(0) = \mathbf{x}_0\\
\label{eqn:boundary2}&\bar{\mathbf{q}}(t) = \mathbf{q}_T, \text{if } t \geq T
\end{align}

This optimization searches for the duration $T$ of the rising motion and the trajectory of the desired joint configuration $\bar{\mathbf{q}}(t)$ to maximize a task-related fitness function $V_{ctrl}$, and subject to physical constraints (eq.(\ref{eqn:dyn1}) and (\ref{eqn:dyn2})) and boundary conditions (eq.(\ref{eqn:boundary1}) and (\ref{eqn:boundary2})). $\mathbf{x}_0$ is the initial condition, and $\mathbf{q}_T$ is the final pose (Figure \ref{fig:task}), both of which are provided by the user. Note that although we can specify the global translation $\mathbf{p}_0$, rotation $\mathbf{r}_0$ and joint angles $\mathbf{q}_0$ in the initial condition, we can only specify the desired joint angles for the final pose because the global translation and rotation are determined by the physical simulation.

Assuming no interbody collision happens when the robot executes $\bar{\mathbf{q}}(t)$, in our tasks, the robot can alway reach the final pose $\mathbf{q}_T$ within a small error due to the weight of the bodies that each actuator supports. The criterion of success for all the tasks is whether the robot remains upright at the end of its motion. We use the following fitness function to reward controllers that keep balance throughout the entire motion.

\begin{equation}
  V_{ctrl}(\mathbf{x}(t))=\int_0^{T+1} \frac{1}{\alpha(t)+\epsilon}\mathrm{d}t
  \label{eqn:controllerObj}
\end{equation}
where $\alpha(t)$ is the angle between the up direction in the local frame of the robot's torso and the up direction in the global frame $(0,0,1)$. It measures how far the robot is to losing its balance. $\epsilon$ is a small positive number to prevent the denominator from being zero. We choose $\epsilon=0.1$ in all our tasks. Note that the upper limit of the integration is $T+1$. The extra one second is to wait the robot to settle down. We use the time horizon $T+1$ because it is still possible that the robot can fall during the settling down phase and our fitness function will penalize this situation.

Two difficulties remain to solve the above optimization. First, the size of the optimization is large, which makes it computationally expensive to solve. Since our robot has 18 degrees of freedom and a rising motion can take a few seconds, the above space-time optimization problem can easily have hundreds to thousands of variables. Instead of directly searching this high dimensional space, we parameterize the controllers to make the computation tractable. Although there are many ways that we can parameterize the control space, designing the most effective control parametrization is not the focus of this work. In fact, we intentionally choose to use a simple parametrization to highlight the effect of our simulation calibration. We use a sparse set of keyframes $\bar{\mathbf{q}}_1, \bar{\mathbf{q}}_2, ..., \bar{\mathbf{q}}_n$ to parameterize the trajectory of the desired poses $\bar{\mathbf{q}}(t)$. In between the keyframes, we linearly interpolate the poses from two adjacent keyframes. With this simplification, the \emph{control parameters} that we need to optimize reduce to only a few keyframes and the time interval between adjacent keyframes. We further halve the size of the problem by exploiting the symmetry of the motion. We find that all three tasks can be achieved with symmetric motions. Thus we constrain that the joint motions on the left bodies mirror those on the right bodies. In addition, since we do not want the robot to use its hands to help standing up, we freeze the joints at the shoulders and the elbows. This focus the controller to the motions of the lower body, which further reduces the size of the optimization problem.

The second difficulty is that during the motion, discrete contact events can happen frequently. They invalidate the gradient information, which imposes additional challenges for the continuous optimization algorithms. We choose to use Covariance Matrix Adaptation (CMA) \cite{Hansen:2009} to optimize the control parameters. Starting from an initial Gaussian distribution, CMA samples this distribution for a set of control parameters, evaluates them using physical simulations, discards the inferior samples and updates the distribution according to the remaining good samples. With a number of iterations, the distribution moves and shrinks, and eventually converges to a good controller parameter that can successfully fulfill the task in the simulation.

\section{Simulation Calibration}

Although the optimal controllers $\bar{\mathbf{q}}(t)$ can work effectively in the simulation, they may fail to achieve the tasks when used on the robot due to the Reality Gap. We develop a simulation calibration subsystem, whose goal is to narrow down the Reality Gap and thus significantly increases the chance that the controller optimized in the simulation can be transferred to the real robot. In this subsystem, we formulate an optimization (eq.~(\ref{eqn:calibration})) that searches for the \emph{simulation parameters} $\mathbf{\theta}$ so that the \emph{discrepancy} $E_{cali}$ is minimized between the simulated results and the robot performance in the real environment.

\begin{equation}
 \min_{\mathbf{\theta}} E_{cali}
\label{eqn:calibration}
\end{equation}

Many parameters need to be set before a physical simulation starts, for example, the mass, the moment of inertia, the COM of each body, the coefficient of restitution and friction, and the gains of the actuators. The accuracy of a physical simulation highly relies on the correctness of these parameter settings because changing simulation parameters can drastically alter the simulation results. Usually, simulation parameters can be set according to the specification sheet of the robot. However, we find that many of these parameters are incorrect. For example, from the CAD file, we calculate the total mass of the robot to be less than 1.1kg, but our own measurement using a scale reads 1.5kg. Furthermore, the height of the COM differs more than 1cm between CAD file and our measurement. This difference in parameters could be due to the manufacturing errors and the weight of cables, glues, nuts and bolts that were used in assembling the robot. Instead of trusting these simulation parameters, we decide to adjust them during simulation calibration.

We improve the simulation accuracy by minimizing the discrepancy $E_{cali}$, which is defined as the difference between the state trajectories in the simulation and those collected in the real robot experiment when the same controller is used in both scenarios. Recall that our entire algorithm is an iterative process. At the $n$th iteration, the optimal controller $\bar{\mathbf{q}}_n(t)$ produces the state trajectory $\mathbf{x}_n(t)$ in the simulation and $\tilde{\mathbf{x}}_n(t)$ on the real robot\footnote{we use the average of the multiple trajectories as $\tilde{\mathbf{x}}_n(t)$ in the robot experiment because even with the same controller, we can get slightly different trajectories due to the varied initial conditions, the noise from the sensor, from the actuator and from the environment.}. Together with the $n-1$ pairs of optimal controllers $\{\bar{\mathbf{q}}_i(t)\}_{i=1,...,n-1}$ and their associated state trajectories $\{\bar{\mathbf{x}}_i(t)\}_{i=1,...,n-1}$ from previous iterations, we compute the discrepancy using the following expression. 

\begin{equation}
  E_{cali}=\frac{1}{n}\sum_{i=1}^{n}\int_{0}^{T+1}||\tilde{\mathbf{x}}_i(t)-\mathbf{x}_i(t)||_{\mathbf{W}}^2\mathrm{d}t
  \label{eqn:calibrationObj}
\end{equation}
where $\mathbf{W}$ is a diagonal weight matrix, which encapsulates the relative importance of each joint. Due to the complex interplay between the simulation results and the simulation parameters, the optimization (\ref{eqn:calibration}) is nonlinear and nonconvex. Similar to controller optimization, we choose to use CMA as the optimization solver. In this case, each CMA sample is a candidate set of simulation parameters $\mathbf{\theta}$. To evaluate each CMA sample, we set the parameters $\mathbf{\theta}$ into the physical simulator, execute the controllers $\{\bar{\mathbf{q}}_i(t)\}_{i=1,...,n}$ to simulate the robot motions $\{\mathbf{x}_i(t)\}_{i=1,...,n}$, and then compute the objective function eq. (\ref{eqn:calibrationObj}).


\section{Results}

\begin{figure*}[!t]
  \centering
  \includegraphics[width=0.95\textwidth]{figures/sit2Stand}
  \caption{The results of the sit-to-stand task in the simulation and on the real robot.}
  \vspace{-0.1in}
  \label{fig:sit2Stand}
\end{figure*}

We evaluate our system using four tasks. Please watch the accompanying video for the robot performance in the simulation and in the real world.

\subsection{Experiment Setup}

We use BIOLOID GP as our robot platform. This dynamic system has totally 22 degrees of freedom. While 18 of them are controllable by Dynamixel AX-12/AX-18 servos, the other six are underactuated, including the root position and orientation. To control the robot, a master program on the PC writes the reference pose $\bar{\mathbf{q}}$ to the serial port that is connected to the robot. A slave program that runs on the robot's onboard microprocessor listens to this port and sends the reference joint angle to each actuator. At the same time, the robot performance data $\tilde{\mathbf{x}}$ is measured and sent back to the computer. We use onboard rotary encoders to measure the joint angles and a VICON motion capture system to measure the global position and orientation of the robot's torso.

Our system is implemented in C++ and runs on a laptop with a 2.6GHz quad-core CPU. We use DART to simulate the physics. We find that all the tasks can be achieved with symmetric lower body motions, which enables us to reduce the dimensionality of the control space in trajectory optimization. For each simulation calibration, we collect three episodes of robot data by running the same reference trajectory three times to average out the noise and other possible perturbations. Each episode is approximately two seconds. We use 32 samples per iteration and at most 50 iterations in CMA. It takes about 15 minutes to find an optimal solution in trajectory optimization or simulation calibration.

\begin{figure}[!b]
  \centering
  \includegraphics[width=0.45\textwidth]{figures/lean2Stand}
  \caption{The results of the lean-to-stand task in the simulation and on the real robot.}
  \vspace{-0.1in}
  \label{fig:lean2Stand}
\end{figure}

\subsection{Rising from a Sitting Position}

The first task is to rise from a chair (Figure \ref{fig:sit2Stand}). The initial and the final poses $\mathbf{q}_0$ and $\mathbf{q}_T$ are shown in the leftmost and rightmost images in Figure \ref{fig:sit2Stand}. The trajectory optimization needs to search for an additional inbetween keyframe $q_1$, as well as the two time intervals $t_1$ and $t_2$ between these keyframes. 

We intentionally choose the initial pose that the feet of the robot are far beyond the projection of the robot's COM in the vertical direction. If the robot simply extends the hips and the knees, it will fall backwards. This is a good test for dynamic balance. Our system successfully finds a reference trajectory that enables the robot to stand up in the simulation: the robot first accumulates a forward momentum by quickly leaning its upper body to the front. It then starts to extend the hips and the knees at the moment when the COM is moving towards the supporting feet.

When we apply this trajectory to the real robot, it works directly, without the need of simulation calibration. This shows that the Reality Gap is not always a problem. In some tasks, the stability region is so large that it can make the discrepancy between the virtual and the real world less critical.

\subsection{Rising from a Leaning Position}

In this task, the robot needs to rise from leaning on the wall to a standing position (Figure \ref{fig:lean2Stand}). The hip joints are initially bent and straightened out in the final configuration while all other joints do not move. The initial and the final poses are the only two keyframes for this task.

\begin{figure}[!b]
  \centering
  \includegraphics[width=0.3\textwidth]{figures/simRobotCompare}
  \caption{Comparisons of the robot's global orientation over time in the simulation (before/after calibration) and in the real environment.}
  \vspace{-0.1in}
  \label{fig:simRobotCompare}
\end{figure}


\begin{figure*}[!t]
  \centering
  \includegraphics[width=0.95\textwidth]{figures/kneel2Stand}
  \caption{The results of the kneel-to-stand task in the simulation and on the real robot.}
  \vspace{-0.1in}
  \label{fig:kneel2Stand}
\end{figure*}

\begin{figure*}[!t]
  \centering
  \includegraphics[width=0.95\textwidth]{figures/stand2Hand}
  \caption{The results of the stand-to-handstand task in the simulation and on the real robot.}
  \vspace{-0.1in}
  \label{fig:stand2Hand}
\end{figure*}

The goal of trajectory optimization is to find an appropriate time interval $T$ between these two keyframes. If the time interval is too long, the robot cannot accumulate enough momentum to rise. If this time interval is too short, the robot can bounce off the wall too quickly and fall forward. Without simulation calibration, the trajectory optimization cannot find a successful trajectory for this task: The robot cannot rise when $T > 0.11s$ and overshoots when $T \leq 0.11s$. Nevertheless, we apply the trajectory with the highest fitness value $T=0.11s$ on the robot. In contrast to the simulation result, in which the simulated robot rises too quickly and falls forward, the robot in the real world actually cannot rise. Figure \ref{fig:simRobotCompare} compares the trajectories of the robot's global orientation in the simulation (blue curve) and in the real world (red curve). After one iteration of simulation calibration, the discrepancy is greatly reduced (Figure \ref{fig:simRobotCompare} green curve). We optimize the trajectory again in this calibrated simulator. This time, the optimal reference trajectory works both in the simulated and in the real world. We successfully cross the Reality Gap with one iteration of simulation calibration, which only needs about 6 seconds of robot data.



\ignorethis{
\begin{figure}[!b]
  \centering
  \includegraphics[width=0.4\textwidth]{figures/fitnessLandscape}
  \vspace{-0.1in}
  \caption{Comparisons of the fitness landscape as more iterations of simulation calibration are performed.}
  \label{fig:fitnessLandscape}
\end{figure}


To better understand how the Reality Gap is gradually narrowed by simulation calibration over multiple iterations, we perform an additional evaluation. Figure \ref{fig:fitnessLandscape} shows how the fitness landscape in the simulation changes with different number of iterations of simulation calibration. The blue curve is the ground truth. It is evaluated on the real robot by varying the time of the last keyframe $T$ in the range of $[0, 0.11]$. The fitness value is calculated according to eq. (\ref{eqn:controllerObj}). The fitness landscape stays at a high value when $T\in[0, 0.1]$, which means that the real robot can successfully rise if the trajectory uses less than 0.1s to change the pose from the initial to the final configuration. In contrast, without simulation calibration, the fitness landscape (lowest black curve) stays at a low value for the entire control space. In other words, no trajectory exists that can make the robot stand up in the simulation. The gap between the blue and the black curves is analogue to the Reality Gap. One iteration of simulation calibration brings the fitness landscape in the simulation towards the ground truth. As more iterations are performed, the fitness landscape in the simulation (brown and red curves) gradually approaches the ground truth, and the Reality Gap is narrowed in this process. Note that a large discrepancy still exists in the region of the parameter space where $T<0.02s$. This is probably caused by two reasons. First, in the region of $T<0.02s$, the torque output of the servo is at its limit but the torque limit is not considered in simulation calibration. Second, the trajectory and the data (the red circles in Figure \ref{fig:fitnessLandscape}) that we use in simulation calibration concentrate on the right half of the parameter space, which makes it difficult to generalize to a region where the data is scarce ($T<0.02$). However, this is beneficial in our applications because the computational resource is focused at the important regions near the successful reference trajectories. This explains why our system requires only a small number of robot experiments.
}

\subsection{Rising from a Kneeling Position}

Figure \ref{fig:kneel2Stand} shows that the robot stands up from a kneeling pose. Between the user-specified initial and final poses, the optimal trajectory consists of two additional keyframes. It demonstrates an agile getting-up motion in the simulation: The robot first leans its upper-body backwards. As its COM is moving to the back, it quickly bends the hip, flexes its ankles and stands up. This entire motion resembles one of the most agile ways that we human get up from a kneeling position when we do not use our hands for additional support. Although this trajectory works perfectly in the simulation, the robot falls backward in the real world. After simulation calibration, we optimize a new trajectory, with which the robot can successfully stand up from in the real world.

\subsection{Flipping to a Handstand Position}
We test our system with a challenging gymnastic action: flipping to a handstand position from a standing pose (Figure~\ref{fig:stand2Hand}). There are two unique challenges in this task. First, the speed and the curvature of the initial arching motion is crucial and only a narrow range of such speed and curvature can lead to a balanced handstand. Second, the USB cable that connects the robot to the computer will inevitably hit the ground during the backflip, which injects a strong perturbation that is not modeled in our simulation.

With two iterations of simulation calibration and trajectory optimization, our system finds a successful trajectory that works both in the simulation and in the real world: The robot arches back rapidly and lifts its feet after the arms touch the ground. It shows that our system can automatically design open loop reference trajectories for challenging tasks, even with strong unmodeled perturbations.


\section{Conclusion}

This section has presented an end-to-end solution to automatically design locomotion controllers for robots. This solution consists of a set of computational tools: a simulation tool that simulates the dynamics of the robot and its environment, an optimization tool that automatically searches for a controller in the virtual environment and a calibration tool that improves the simulation accuracy to ease controller transfer from the virtual to the real world. This powerful system allows us to efficiently design controllers of a humanoid robot to achieve three different tasks, rising from leaning, sitting and kneeling poses to an erect stance.

Since the main goal of this work is to demonstrate that the computational tools developed for character animation, including physical simulation and controller optimization, can be applied to robotics, the biggest challenge is to cross the Reality Gap. The evaluation shows that our simulation calibration algorithm is effective to narrow down this gap. In all the examples, at most one iteration of calibration is needed before we can transfer the controller to the real robot. However, we want to emphasize that the goal of simulation calibration is not to find the  true simulation parameters. Instead, it finds a set of parameters that reduce the discrepancy between the simulation and the real experiment for a specific locomotion task. We observe that the simulation parameters optimized for the task of lean-to-stand can be entirely different from those of kneel-to-stand. In other words, the calibrated simulator is only valid for the current task and should not be used in a different task. If the tasks are the same but the initial configuations are slightly different, the result of simulation calibration can be reused. For example, in the task of lean-to-stand, we can use a simulator calibrated for a specific initial leaning angle to optimize controllers if the robot needs to stand up from a different initial leaning angle. In this case, our experiments show that the result of simulation calibration can be reused if the initial leaning angle is perturbed within five degrees. 

It would be more interesting if simulation calibration can discover the truth so that the calibrated simulator can be used for different tasks. We believe that this is possible if we use controllers and their corresponding robot performance data from multiple different tasks as the input of simulation calibration. We leave this as one of the future work. In our examples, we have shown that adjusting the COM and the actuator gains are enough, but other simulation parameters might also be important for a wider range of tasks. Including more simulation parameters and performing feature selection would be a promising direction for future work. In addition, some discrepancies between the simulation and the real world may not be explained by inaccurate simulation parameters alone. Unmodeled dynamics could also contribute to the discrepancy. Combining parametric and non-parametric models in simulation calibration for unmodeled dynamics is also an intereting avenue for future work.

\bibliographystyle{IEEEtran}
\bibliography{icra}
\appendix
\section{Derivation of Actuator Model}

\begin{figure}[t]
\centering
\includegraphics[width=0.45\textwidth]{figures/ax18gain.eps}
\caption{The mapping between $q-\bar{q}$ and $U$ for an AX-18 actuator \cite{AX18:2015}. The x-axis is $q-\bar{q}$ while the y-axis is $U$.}
\label{fig:actuatorMap}
\end{figure}

In this Appendix, we derive the actuator model (eq.(\ref{eqn:torqueErrorRelationSimple})) from the specifications of the Dynamixel AX-18 servo, which use the following mapping (Figure \ref{fig:actuatorMap}) between the joint angle difference $q-\bar{q}$ and the power level $U$. The intervals A and D determine the slope of the actuator response for counter-clockwise and clockwise motions respectively. Smaller values mean steeper response slopes, in which case the actuator follows the desired angle more closely. However, too small a value can lead to overshooting problems. B and C are the compliance margins. If the error of angle is within a small margin specified by B and C, the servo does not output any torque. E, the punch, is the minimum power level before the servo shuts down. In practice, we set A and D to be the same so that the servo will behave the same no matter it rotates clockwise or counter-clockwise. In addition, since B, C and E are very small compared to A and D, we ignore their effects and approximate the mapping as linear within the intervals $q-\bar{q}\in A\bigcup B\bigcup C\bigcup D$ with the slope $k_e$:
\begin{displaymath}
  U=k_e(q-\bar{q})
\end{displaymath}

To derive the relation between the power level $U$ and the output torque $\tau$, we adopt a model for the ideal DC motor \cite{SchwarzB:2013}. It is valid to assume an ideal model because the AX-18 servos use high-quality DC motors. The derivation follows by considering the power balance in the motor at a constant voltage U:
\begin{equation}
  P_{electric} = P_{mechanic} + P_{heat}
  \label{eqn:powerBalance}
\end{equation}
where $P_{electric}$ is the electrical power, $P_{mechanic}$ is the mechanical power, and $P_{heat}$ is the power dissipated as heat. From eq.(\ref{eqn:powerBalance}), we can get the following relation:
\begin{equation}
UI=\dot{q}\tau_{motor} + RI^2
\end{equation}
where $I$ is the current and $R$ is the motor winding resistance. In an ideal DC motor, the torque is linearly proportional to the current $\tau_{motor}=k_{\tau}I$. Plugging it into the above equation, we arrive at the relation between the voltage $U$ and the total torque generated by the motor $\tau_{motor}$:
\begin{equation}
  U=k_{\tau}\dot{q}+\frac{R}{k_{\tau}}\tau_{motor}
  \label{eqn:votageTorqueRelation}
\end{equation}
where $k_{\tau}$ is the torque constant, which is determined by the hardware design of the motor. The total torque generated by the motor is not yet the output torque that drives the motor shaft due to the friction inside the motor. The total torque can be decomposed into the output torque $\tau$ and the friction torque $\tau_f$.
\begin{equation}
  \tau_{motor}=\tau+\tau_f
  \label{eqn:torqueBalance}
\end{equation}
The friction torque can be further divided into viscous friction and Coulomb friction \cite{SchwarzB:2013}:
\begin{equation}
  \tau_f = k_v\dot{q}+k_c\sgn(\dot{q})
  \label{eqn:frictionComponents}
\end{equation}
where $k_v$ and $k_c$ are friction coefficients for the viscous and Coulomb friction respectively. $\sgn(x)$ is the sign function that equals 1 if $x$ is positive, -1 if $x$ is negative and 0 otherwise.

Combining eq.(\ref{eqn:votageTorqueRelation}), (\ref{eqn:torqueBalance}) and (\ref{eqn:frictionComponents}), we get the relation between the error of the joint angle $q-\bar{q}$ and the output torque $\tau$.
\begin{align}
\nonumber  \tau & = \frac{k_{\tau}k_e}{R}(q-\bar{q})+(-k_v-\frac{k_{\tau}^2}{R})\dot{q}-k_c\sgn(\dot{q})\\
\nonumber & = -k_p(q-\bar{q}) - k_d\dot{q} - k_c\sgn(\dot{q})
\end{align}
where $k_p=-\frac{k_{\tau}k_e}{R}$ and $k_d=k_v+\frac{k_{\tau}^2}{R}$. 


\end{document}
