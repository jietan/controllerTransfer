\section{Related Work}

The locomotion tasks that this paper foucses on involve changing between two poses and maintaining balance during the motion. Two tasks that falls into this category and are extensively studied in robotics are sit-to-stand \cite{Faloutsos:2003,Iida:2004,Pchelkin:2010,Mistry:2010,Bahar:2014} and lie-to-stand \cite{morimoto:1998,Faloutsos:2001,Kanehiro:2007,Hirukawa:2005}. While many of these prior work focuses on one specific motion, we target at a more general problem of finding controllers for a wide range of such locomotion tasks. A few related work tried to tackle this more general problem. Jones \cite{jones:2011} developed rising motions for both biped and quadraped using pose tracking, orientation correction and virtual force. Lin and Huang \cite{lin:2012} used motion planning and dynamics filtering to develop rising up motions from various initial lying poses. Tassa et al. \cite{tassa:2012} used Model Predictive Control to synthesize complex behaviors, including getting up from an aribitrary pose on the ground. Although these work has shown impressive results in simulation, experiments on real robots were not presented.

Locomotion often involve frequent change of contacts. It poses significant challenges to controller optimization due to the discontinuous contact forces. We choose to use Covariance Matrix Adaptation (CMA) \cite{Hansen:2004}, a stochastic sampling-based optimization algorithm, to tackle this challenge. Although CMA is not widely used in robotics, it is a popular method in physically-based character animation to search for control parameters when the problem domain is highly discontinous \cite{Wu:2010, Wang:2010, Tan:2011}.

A controller that is designed in simulation may not work in the real environment. This discrepancy is called Reality Gap. One way to cross the reality gap is to improve the simulation model using real data measured from robot experiments. 

system identification need a lot of data because the task used in system identification is often different from the real task. Real task involves contacts, which makes system identification difficult. CMA is used to optimize gait in computer animation, which should be able 

. The simulation is improved by measuring and minimizing the discrepancy between the simulation results and the data collected in robot experiments. Ha and Yamane \cite{HA:2015} modeled this discrepancy using Gaussian process. Abbeel et al. \cite{Abbeel:2006} used an inaccurate physical model but successively grounded the policy evaluations using real-life trials.Grounded simulated learning approach \cite{Farchy:2013} iteratively optimized the controller, measured the discrepancy and modified the simulator using supervised learning algorithms. Bongard and Lipson \cite{BongardL05} coevolved the controller and the simulator using an iterative estimation-exploration process. Similarly, Zagal et al. \cite{zagal2004} introduced the ``back-to-reality'' approach, which also involved the coevolution but used a different measure of discrepancy.
