\section{Simulation Calibration}

Although the optimal controllers $\bar{\mathbf{q}}(t)$ can work effectively in the simulation, they may fail to achieve the tasks when used on the robot due to the Reality Gap. We develop a simulation calibration subsystem, whose goal is to narrow down the Reality Gap and thus significantly increases the chance that the controller optimized in the simulation can be transferred to the real robot. In this subsystem, we formulate an optimization (eq.~(\ref{eqn:calibration})) that searches for the \emph{simulation parameters} $\mathbf{\theta}$ so that the \emph{discrepancy} $E_{cali}$ is minimized between the simulated results and the robot performance in the real environment.

\begin{equation}
 \min_{\mathbf{\theta}} E_{cali}
\label{eqn:calibration}
\end{equation}

Many parameters need to be set before a physical simulation starts, for example, the mass, the moment of inertia, the COM of each body, the coefficient of restitution and friction, and the gains of the actuators. The accuracy of a physical simulation highly relies on the correctness of these parameter settings because changing simulation parameters can drastically alter the simulation results. Usually, simulation parameters can be set according to the specification sheet of the robot. However, we find that many of these parameters are incorrect. For example, from the CAD file, we calculate the total mass of the robot to be less than 1.1kg, but our own measurement using a scale reads 1.5kg. Furthermore, the height of the COM differs more than 1cm between CAD file and our measurement. This difference in parameters could be due to the manufacturing errors and the weight of cables, glues, nuts and bolts that were used in assembling the robot. Instead of trusting these simulation parameters, we decide to adjust them during simulation calibration.

We improve the simulation accuracy by minimizing the discrepancy $E_{cali}$, which is defined as the difference between the state trajectories in the simulation and those collected in the real robot experiment when the same controller is used in both scenarios. Recall that our entire algorithm is an iterative process. At the $n$th iteration, the optimal controller $\bar{\mathbf{q}}_n(t)$ produces the state trajectory $\mathbf{x}_n(t)$ in the simulation and $\tilde{\mathbf{x}}_n(t)$ on the real robot\footnote{we use the average of the multiple trajectories as $\tilde{\mathbf{x}}_n(t)$ in the robot experiment because even with the same controller, we can get slightly different trajectories due to the varied initial conditions, the noise from the sensor, from the actuator and from the environment.}. Together with the $n-1$ pairs of optimal controllers $\{\bar{\mathbf{q}}_i(t)\}_{i=1,...,n-1}$ and their associated state trajectories $\{\bar{\mathbf{x}}_i(t)\}_{i=1,...,n-1}$ from previous iterations, we compute the discrepancy using the following expression. 

\begin{equation}
  E_{cali}=\frac{1}{n}\sum_{i=1}^{n}\int_{0}^{T+1}||\tilde{\mathbf{x}}_i(t)-\mathbf{x}_i(t)||_{\mathbf{W}}^2\mathrm{d}t
  \label{eqn:calibrationObj}
\end{equation}
where $\mathbf{W}$ is a diagonal weight matrix, which encapsulates the relative importance of each joint. Due to the complex interplay between the simulation results and the simulation parameters, the optimization (\ref{eqn:calibration}) is nonlinear and nonconvex. Similar to controller optimization, we choose to use CMA as the optimization solver. In this case, each CMA sample is a candidate set of simulation parameters $\mathbf{\theta}$. To evaluate each CMA sample, we set the parameters $\mathbf{\theta}$ into the physical simulator, execute the controllers $\{\bar{\mathbf{q}}_i(t)\}_{i=1,...,n}$ to simulate the robot motions $\{\mathbf{x}_i(t)\}_{i=1,...,n}$, and then compute the objective function eq. (\ref{eqn:calibrationObj}).

In our work, we initialize the simulation parameters as follows. We set the physical properties of each body, including the mass, the moment of inertia and the COM according to the CAD files. We set the actuator gains based on the measurement from experiments (Chapter \ref{sec:motorDynamics}). We leave all other parameters as default values in DART. Although these parameters are not accurate, they serve as a good initial guess. During simulation calibration, we search the parameter space within a bounded range centered at the initial guess. In addition, we also employ two simplifications to speed up the optimization. First, we manually select the most relevant simulation parameters that need to be optimized. All our tasks are to achieve a specific final pose while keeping balance. Since accurate actuator gains determine whether the robot can reach and hold the final pose, and correct COM's play an important role in balance control, we decide that the actuator gains of servos and the COM of each body are the most important simulation parameters in our case. This manual selection drastically reduce the search space of the optimization. Second, we zero out most of the diagonal entries of $\mathbf{W}$ in eq. (\ref{eqn:calibrationObj}) except for the rows corresponding to the global orientation. More specifically, we measure the discrepancy based solely on $\alpha$, the angle between the up direction in the local frame of the robot's torso and the up direction in the global frame $(0,0,1)$.

\begin{equation}
  E_{cali}=\frac{1}{n}\sum_{i=1}^{n}\int_{0}^{T+1}(\tilde{\alpha}_i(t)-\alpha_i(t))^2\mathrm{d}t
  \label{eqn:calibrationObj1}
\end{equation}

The objective function eq. (\ref{eqn:calibrationObj1}) captures the most important features that characterize the success or failure of our tasks (see eq.(\ref{eqn:controllerObj})), and eliminates the tedious manual tuning of the weight matrix $\mathbf{W}$.