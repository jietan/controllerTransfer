\section{Overview}

\begin{figure}[!t]
  \centering
  \includegraphics[width=0.5\textwidth]{figures/controllerTransfer}
  \caption{Overview of our algorithm.}
  \vspace{-0.1in}
  \label{fig:controllerTransferOverview}
\end{figure}

We have developed a system that can automatically design reference trajectories for robots to execute transition motions (Figure~\ref{fig:controllerTransferOverview}). Given the specification of the robot, including its body shape, the physical properties of each body, and the types of joints, we build a physical simulation using Dynamic Animation and Robotics Toolkit (DART) \cite{dart:2012}. The trajectory optimization subsystem runs thousands of simulations to search for the optimal trajectory that maximizes the task-related fitness function. We then test this optimal trajectory on the robot. If the robot successfully completes the task, a solution is found and our algorithm terminates. Otherwise, we record the robot performance data and feed it into the simulation calibration subsystem. Simulation calibration runs another optimization, which searches for the optimal simulation parameters to minimize the discrepancy between the performance of the robot in the simulation and in the real world. The loop of trajectory optimization and simulation calibration is performed iteratively until the reference trajectory works successfully on the real robot. In the next three sections, we will present the algorithmic details of these components.
